\newcommand{\primarymetric}{62,0 \%}

\begin{prefacesection}{Abstract}
    Die Erkennung von Aktionen in Videos (auch \gls{har}) ist Gegenstand aktueller Forschung im Bereich Video Understanding.
    Großspurige Datensets beschränken sich meist auf allgemeine, Domänen-übergreifende Klassen, die eindeutig voneinander abgrenzbar sind und nicht in Kombination auftreten.
    Hingegen finden im Bereich von Feldsportarten, wie Fußball, oft viele, visuell ähnliche Aktionen in kürzester Zeit statt.
    Bestehende Datensets dieser Domäne bilden bislang nur einen Bruchteil der Spielaktionen aus der realen Welt ab.
    Hier knüpft die vorliegende Arbeit an, in der ein Multi-Label-Datenset mit insgesamt 32 verschiedenen Spielaktionen, basierend auf TV-Aufzeichnungen von Fußballspielen, neu generiert wird.
    In anschließenden Benchmarks kommen verschiedene \gls{dnn}-Architekturen (wie R2+1D, SlowFast, ir-CSN) als Video-Classifier zum Einsatz, die anhand des neuen Datensets mit Methoden des Transfer-Learnings und ohne weitere Zwischenschritte (\emph{Ende-zu-Ende}) trainiert werden.
    Die Ergebnisse zeigen, dass aktuelle Modelle des Deep-Learnings bereits in der Lage sind, diese Aktionen ausreichend gut zu erkennen (mit einer Balanced Accuracy von \primarymetric).
    Zum anderen können Spielaktionen in verschiedene Schwierigkeitsgrade kategorisiert werden, deren Erkennung maßgeblich abhängt von der Auflösung und der Geschwindigkeit der Videos.

    \begin{tcolorbox}[title=Todo]
        Ergebnisse und Eigenschaften des Datensets, Englisch nochmal übersetzen
    \end{tcolorbox}


    Action Recognition is the key problem of Video Understanding and has not been explored as far as the two-dimensional variant of Object Recognition.
    However, recently there have been frequent new breakthroughs that replace existing models.
    Videos of field sports, such as soccer, are particularly demanding in this area, since possible game actions have a high intra-class variance and low inter-class similarity.
    Existing data sets only represent a fraction of game actions of the real world and are hardly relevant in the field of soccer scouting.
    This is where the work at hand follows, the aim of which is to train a central multi-label video classifier for recognizing game actions with a data set specially generated for this purpose.
    This data set offers a spectrum of 32 different classes, \dots
    The results should benefit from the training of a large number of classes as well as the use of factorized 3D convolution.

\end{prefacesection}