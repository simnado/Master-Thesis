\chapter{Paarweise Intersektion von Spielaktionen}
\label{ch:overlaps}

In \autoref{fig:overlaps} wurde für alle 32 Klassen der prozentuale Anteil aller Sampler einer Klasse berechnet, der jeweils in Kombination mit einer anderen Klasse auftritt.
Ein Eintrag $c_{ij}$ lässt sich wie folgt interpretieren:
$c_{ij}$ \% aller Samples der Klasse $i$ treten gemeinsam mit der Klasse $j$ auf.

Wie man sieht, ereignen sich \zB 74 \% der Aktionen \code{saved} gleichzeitig mit Aktionen der Klasse \code{footShot}.
Bei einer niedrigeren Wahl von $\Delta$ sind die Abhängigkeiten entsprechend geringer, es treten jedoch die gleichen Kombinationen hoher Abhängigkeiten auf.

\begin{figure}
    \centering
    \bigimage{img/05_pairwise_occurrences_all_delta_5.eps}{\textwidth}
    \caption[Anteil paarweiser Überschneidungen pro Klasse]{Anteil paarweiser Überschneidungen pro Klasse bei Segmentierung aller Halbzeiten mit $\Delta = 5$ (Quelle: Eigene Darstellung)}
    \label{fig:overlaps}
\end{figure}
