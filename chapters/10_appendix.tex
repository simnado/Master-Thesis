\chapter{Rechenergebnisse von HAR-Modellen in der Literatur}
\label{ch:leaderboard}

\autoref{tab:models} zeigt die publizierten Rechenergebnisse der in \autoref{ch:sota} vorgestellten Modelle im Vergleich.
Die Ergebnisse von UCF-101 beziehen sich alle auf die als 3-fold-Accuracy angegebene Metrik.
Sports-1M zeigt jeweils die Accuracy im Format \emph{clip@1}/\emph{video@1} und Kinetics stets die Top-1-Accuracy.

\begin{figure}
    \csvautotabular{tbl/models.csv}
    \label{tab:models}
\end{figure}

\chapter{Aktionskatalog}
\label{ch:aktionskatalog}

Eine übersichtliche Auflistung aller erfassten Aktionen ist in \cite{StatsbombDocs16} zu finden.


In \autoref{tab:action} werden alle 32 Klassen des im Rahmen dieser Arbeit generierten und in \autoref{ch:data} vorgestellten Datensets aufgelistet.
Neben der Anzahl ist die durchschnittliche und maximaler Dauer pro Aktion vermerkt, die sich aus den \gls{annotationen} in \cite{Statsbomb20} ergeben.
Aktionen, die hier keinen Wert vorweisen werden lediglich als Zeitpunkt in \gls{sbod} repräsentiert.

\begin{figure}
    \centering
    \begin{subfigure}{0.45\textwidth}
        \centering
        \csvautotabular{tbl/actions_a.csv}
    \end{subfigure}
    \begin{subfigure}{0.45\textwidth}
        \centering
        \csvautotabular{tbl/actions_b.csv}
    \end{subfigure}
    \label{tab:action}
\end{figure}

\chapter{Paarweise Intersektion von Spielaktionen}
\label{ch:overlaps}

In \autoref{fig:overlaps} wurde für alle 32 Klassen der prozentuale Anteil aller Sampler einer Klasse berechnet der jeweils in Kombination mit einer anderen Klasse auftritt.
Ein Eintrag $c_{ij}$ lässt sich wie folgt interpretieren:
$c_{ij}$ \% aller Samples der Klasse $i$ treten gemeinsam mit der der Klasse $j$ auf.

Wie man sieht, ereignen sich \zB 74 \% der Aktionen \code{saved} (innerhalb eines Zeitkontextes von $\Delta=4$ Sekunden) gleichzeitig mit Aktionen der Klasse \code{footShot}.
Bei einer niedrigeren Wahl von $\Delta$ sind die Abhängigkeiten entsprechend geringer, es treten jedoch die gleichen Kombinationen hoher Abhängigkeiten auf.

\begin{figure}
    \centering
    \bigimage{img/data-plots/4sec/pairwise_occurrences_all_202010-1419-3817}{\textwidth}
    \caption{Anteil paarweiser Überschneidungen pro Klasse (hier: $\Delta = 4$)}
    \label{fig:overlaps}
\end{figure}
