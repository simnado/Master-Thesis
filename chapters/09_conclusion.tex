\chapter{Fazit}
\label{ch:zusammenfassung}

In diesem letzten Kapitel wird zunächst das Vorgehen und die Erkenntnisse der Arbeit rekapituliert.
Anschließend werden offene Probleme genannt und ein Ausblick für deren Lösung in Ausblick gestellt.
Darüber hinaus werden weiterführende Forschungsfrage ausgeworfen.

\section{Zusammenfassung}
\label{sec:rekapitulation}

Zur \gls{har} gibt es im Fußball-Sektor bis dato kein umfangreiches Datenset mit mehr als zehn Aktionsklassen.
Deshalb wurde im Zuge dieser Arbeit mit SOCC-HAR-32 ein mächtigeres Multi-Label-Datenset mit insgesamt 32 Klassen auf Basis öffentlich zugänglicher Daten erstellt.
Das Datenset selbst besteht aus Markierungen, die sich auf ein Intervall innerhalb eines ungeschnittenes Video beziehen.

Um zu prüfen, ob alle neuartigen Klassen mit Methoden des Deep-Learnings lernbar sind, wurden drei Baseline-Modelle aus drei unterschiedlichen Oberkategorien auf Eignung getestet:
SlowFast, R2+1D und ir-CSN.
Unter allen Baseline-Modelle schnitt ir-CSN bei einem einheitlichen Testset mit dreisekündigen Clips am besten ab.
Anschließend wurden optimale Hyperparameter gesucht, wobei sich herausstellte, dass zusätzliche Frames und eine erhöhte Schrittweite beim Sampling die Ergebnisse zusätzlich verbessern, was eine Segementierung in fünfsekündige Clips zur Folge hat.

Außerdem wurde die Komplexität der einzelnen Aktionsklassen anhand erhobener Metriken verglichen und in eine Rangordnung gebracht.
Dabei wurden besonders schwierig zu erlernende Klassen aus einem abgeleiteten Datenset namens SOCC-HAR-25 entfernt.

Durch erneutes Training mit SOCC-HAR-25 und durch die Behebung zahlreicher Datenfehler konnten weitere Verbesserungen erzielt werden.
Zuletzt wurde eine Steigerung durch das Sampling mit einer höheren Auflösung und zusätzlicher Samples erreicht, was zu einer Balanced Accuracy von \primarymetric \% (32 Klassen) \bzw \secondarymetric \% (25 Klassen) führte.

\section{Weiterführende Erkenntnisse}
\label{sec:weiterfuhrende-erkenntnisse}

Mit Hinblick auf die Forschungsfrage wird mit SOCC-HAR-32 (und SOCC-HAR-25) ein neues Datenset vorgestellt, welches mehr als dreimal (zweimal) so viele Aktionen abbildet, als bisherige Fußball-Datensets im Bereich HAR.
Die Experimente belegen dabei, dass der Großteil der Aktionsklassen mit aktuellen Deep-Learning-Modelle erlernbar ist, wobei sich ir-CSN im Vergleich zu den anderen benutzen Modelle als besonders geeignet herausstellt.
Als besonders schwer gilt weiterhin die Erkennung von Aktionen, die einen hohen Detailgrad oder einen langen zeitlichen Kontext benötigen, um identifiziert werden zu können.

\section{Ausblick}
\label{sec:ausblick}

Im Rahmen dieser Arbeit war es nicht mehr möglich alle Vorhaben in Perfektion umzusetzen.
Bisherige Ergebnisse leiden vor allem an der mangelhaften Datenqualität \bzw der Qualität des Alignings.
Durch eine umfassendere Verifikation und dem manuellen Justieren von Zeitfenstern innerhalb des Datensets wird entsprechend eine große Verbesserung der Ergebnisse zugerechnet.
Als weiterer Schwachpunkt wird die statische Segmentierung in Clips angesehen:
Eine aufwendigere Sampling-Strategie, die sicherstellt, dass Aktionen innerhalb eines Clips immer zeitlich zentriert sind, würde ebenfalls eine potenzielle Verbesserung mit sich ziehen.
Die hier verwendete statische Segmentierung kann \zB nicht sicherstellen, dass nur die erste Hälfte einer Aktion gezeigt wird.
Ebenfalls wird davon ausgegangen, dass zusätzliche Background-Samples während des Trainings zur einer höheren Precison und einem höheren F1-Score führen.

Mit Hinblick auf eine zeitliche Action Detection könnte das Baseline-Modell in eines der in \autoref{sec:temporal-action-detection} vorgestellten Modelle integriert werden.
Für das weiterführende Training eines Action-Detection-Modells ist jedoch auch (und zwar in noch höherem Maße) die Datenqualität der Aktionsintervalle entscheidend.

Eine weitere Fortsetzung wäre die Aktionen mit Spielfeldkoordinaten zu verknüpft:
das Datenset kann auf derselben Datengrundlage (\gls{sbod}) um Spielfeldkoordinaten des ausführenden Spielers (oder des Balls) erweitert werden, sodass als zusätzlicher Output pro Aktion eine Position ermittelt wird.
Die Ergebnisse eines separaten Modells zum Tracking von Spielern könnte mit diesen Koordinaten verknüpft werden, sodass die Aktionen letztlich ausführenden Spielern zugeordnet werden können.
So wäre eine vollständige Erfassung Spieler-bezogenen Statistiken auf Basis von Spielaktionen und Spielerpositionen möglich, die auch eine automatische, taktische Analyse im Fußball-Scouting ermöglichen würde.

\section{Forschungsbedarf}
\label{sec:fazit}

Mit Hinblick auf die hier vorgestellte Baseline-Modell, steht der direkte Vergleich mit den Ergebnissen aus~\cite{Giancola18} und~\cite{Jiang19} noch aus.
Das Baseline-Modell muss dazu mit der gleichen Datengrundlage nachtrainiert und verglichen werden.
Erst dann lässt sich abschließend sagen, ob ir-CSN als \gls{har}-Modell besser geeignet ist, als die dort verwendeten Ansätze.

Analog sind weitere Benchmarks mit alternativen Modellen oder tieferen Varianten der hier verwendeten Modelle notwendig, um abschließend sicherzustellen, dass sich ir-CSN unter den vorgestellten Modellen am besten eignet.
Dazu zählen insbesondere SlowFast-101, aber auch das in \autoref{subsec:factorized-convolution} vorgestellte X3D-L, welches eine noch deutliche höhere Auflösung ermöglicht und in anderen Benchmarks vergleichbar gut abschneidet.
Außerdem können die Ergebnisse abweichen, wenn die Modelle von Grund auf mit SOCC-HAR-32 trainiert werden, statt nur mit reinem Transfer-Learning.

Zuletzt sei noch der Fortschritt durch die Hinzunahme von Audiosignalen genannt.
Ein ähnlicher Ansatz wurde bereits in~\cite{Wang19} in Kombination mit einem SlowFast-Modelle beschrieben.
