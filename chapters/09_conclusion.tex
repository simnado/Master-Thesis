\chapter{Rekapitulation}
\label{ch:zusammenfassung}

In diesem letzten Kapitel wird zunächst das Vorgehen und die Erkenntnisse der Arbeit rekapituliert.
Anschließend werden offene Probleme genannt und ein Ausblick für deren Lösung in Ausblick gestellt.
Darüber hinaus werden weiterführende Forschungsfrage ausgeworfen.

\begin{tcolorbox}[title=WIP]
 \begin{itemize}
  \item Zusammenfassung: Vorgehen, Ergebnisse, Erkenntnisse (Forschungsfrage beantworten, Was zu beweisen war)
  \item Ausblick: distanzierte Bewertung im Kontext der Problemstellung (Schneller durch mehr CPU-Power, Decoding ist primärer Bremser!), tiefere Modelle, action spotting, non-local
  \item Forschungsbedarf: Weiterführung, Anschlussstudien
 \end{itemize}
 \end{tcolorbox}

\section{Zusammenfassung}
\label{sec:rekapitulation}

Zur \gls{har} gibt es im Fußball-Sektor bis dato kein umfangreiches Datenset mit mehr als zehn Aktionsklassen.
Deshalb wurde im Zuge dieser Arbeit zunächst ein mächtigeres Multi-Label-Datenset mit insgesamt 32 Klassen auf Basis öffentlich zugänglicher Daten erstellt.
Das Datenset selbst besteht aus Markierungen, die sich auf ein Intervall innerhalb eines ungeschnittenes Video beziehen.

Um zu prüfen ob alle neuartigen Klassen mit Methoden des Deep-Learnings lernbar sind, wurden drei Baseline-Modelle aus drei unterschiedlichen Oberkategorien auf Eignung getestet:
SlowFast als reines, R2+1D als Block-faktorisiertes und ir-CSN als Gruppen-faktorisiertes 3D-\gls{cnn}.
Für jedes Baseline-Modell wurden optimale Hyperparameter gesucht, wobei alle Modelle bessere Ergebnisse erzielen, wenn sie auf \glspl{clip} mit einer längeren Dauer trainiert werden.
Die zusätzliche Länge kann durch zusätzliche Frames oder durch ein dünneres Sampling erfolgen, wobei ersteres den größeren Erfolg verspricht.
In einem Benchmark wurden die besten Konfigurationen der Baseline-Modelle übernommen und anhand von xxx-sekündigen Samples eines separaten Testsets verglichen, wobei ir-CSN mit XXX am besten abschnitt.

Weitere Verbesserungen des CSN-Modells konnten erzielt werden, indem Samples mit auffällig hohen Fehlerwerten manuell verifiziert wurden.
Durch die manuellen Korrekturen konnte eine Verbesserung von XXX erzielt werden.
Eine weitere Steigerung wurde durch das Sampling mit einer höheren Auflösung, zusätzlicher Samples und Label Smoothing erreicht.

Darüber hinaus wurde auf Basis der Rechenergebnisse eine Analyse über die Schwierigkeit der 32 Aktionsklassen durchgeführt, wobei auffiel, einige Aktionen praktisch gar keine Wiedererkennung erfahren.
Diese Aktionen wurden in Form eines Sub-Datensets mit nur 24 der ursprünglichen 32 Klassen entfernt und die Experimente wurden mit dem neuen Datenset wiederholt, was die höchste Performance-Steigerung brachte.

\section{Weiterführende Erkenntnisse}
\label{sec:weiterfuhrende-erkenntnisse}

Mit Hinblick auf die Forschungsfrage wird mit SOCC-HAR-24 ein neues Datenset vorgestellt, welches 14 zusätzliche Klassen beinhaltet, die, wie anhand der Experimente belegt wurde, durch aktuelle Deep-Learning-Modelle erlernbar sind.

Die Experimente legen weiterhin nah, dass bei einer \gls{har} mit Fußballvideos der Dynamik eine höhere Rolle zukommt als es bei generischen Datensets der Fall ist.
Die höhere Genauigkeit von SlowFast-4x16 gegenüber SlowFast-8x8, sowie das Hyperparameter-Tuning stützen diese Hypothese.

Die Analyse der Aktionsklassen belegt, dass bestiommte Aktionen (die \zB oft in Kombination mit Nahaufnahmen auftreten) vergleichsweise schlecht abschneiden, während andere durchweg besser abschneiden.

\section{Ausblick}
\label{sec:ausblick}

Im Rahmen dieser Arbeit war es nicht mehr möglich einen abschließenden Test der mit SOCC-HAR-32 trainierten Modelle, auf den Daten von SoccerNet und SoccerDB durchzuführen.


Das belegt zwar, dass die Daten in SOCC-HAR-32 nicht frei von Fehlern sind, bedeutet im Umkehrschuss jedoch auch, dass

Die Rechenergebnisse dieser Arbeit stellen eine solide Baseline zukünftiger Arbeiten dar, die durch zusätzlichen Aufwand noch viel Verbesserungspotential hat.
Offensichtlich lassen sich noch bessere Ergebnisse erziehen, wenn das vorgestellte Baseline-Modell von Grund auf mit den neuen Datenset trainiert würde, was aus Hardware-Beschränkungen im Rahmen dieser Arbeit nicht möglich war.
Weitere Fortschritte sind denkbar durch die Hinzunahme der Audiosignale der zugrunde liegenden Videos.
Ein ähnlicher Ansatz wurde bereits in \cite{Wang19} beschrieben.
Ebenso vielversprechend ist die in \cite{Wu20} vorgestellte Methodik die räumliche Auflösung während des Trainings von niedrig- zu höher auflösenden Samples zu variieren.

Mehrschrittige Modell, die aufgrund des erhöhten Entwicklungsaufwand im Vorfeld ausgeschlossen wurden, könnten ebenfalls Fortschritte bringen.
So könnten \zB vorab Kamerawechsel erkannt werden und \glspl{clip} ausschließlich so gesamplet werden, dass sie kontinuierliche Bewegung ohne Kameraschnitte zeigen.
Alternative kann die Information eines Kamerawechsels, wie auch weitere separat erhobene Informationen wie Ballposition oder Spielerkoordinaten in das Baseline-Modell einfließen.

Im Anschluss an die \gls{har} kann die hier nur naiv umgesetzte zeitliche Action Detection durch den Einsatz eines mächtigeren Modells aus \autoref{sec:temporal-action-detection} ersetzt werden, die mit genaueren \gls{annotationen} eines abgewandelten Datensets trainiert werden können.

Zusätzlich kann auf der selben Datengrundlage (mit gleichen Spielen) ein Datenset mit Spielerkoordinaten und Identifiern erstellt werden.
Mit solchen Daten kann \ggf ein Modell zum Tracking von Spielern oder sogar der Projizierung von Spielern auf Spielfeldkoordinaten erstellt werden und mit dem hier vorgestellten Modell zur Aktionserkennung verknüpft werden.
So wäre eine vollständige Erfassung Spieler-bezogenen Statistiken auf Basis von Spielaktionen und Spielerpositionen möglich, die das Fußball-Scouting nachhaltig prägen könnte.

schlecht:
Zur Laufzeit müssen diese ungeschnittenen Videos in kleinere \glspl{clip} einer fester Länge segmentiert werden.
Anhand der zugrundeliegenden Stammdaten, wurde eine Minimaldauer von 3 und eine Maximaldauer von 6 Sekunden für diese Segmentierung festgelegt.
Die Segmentierung des Trainings- und Validierungssets folgt

* Datenset Format anders: Superset von Soccernet, SoccerDB

\section{Forschungsbedarf}
\label{sec:fazit}

Abschließend lassen sich die Fragestellungen aus \autoref{sec:forschungsfrage} beide mit \emph{Ja} beantworten.
Die Lernbarkeit von bis zu 32 wurde im Zuge zahlreicher Experimente mehrmals bewiesen.
Als besonders anspruchsvoll gelten hierbei \dots

% todo: zweite frage

Auch der produktive Einsatz der \gls{har} in eine Anwendung zur zeitlichen Action Detection war möglich.
Aufgrund zu ungenauer Intervallgrenzen in dem neu erhobenen Datenset, konnte der Erfolg der Detection nicht quantifiziert werden.
Jedoch werden die Ergebnisse (subjektiv) für anwendbar befunden.
