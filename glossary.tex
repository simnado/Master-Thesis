\newglossary[tlg]{symbols}{tld}{tdn}{Symbol Verzeichnis}

\makeglossaries

%%%%%%%%%%%%%%%%%%%%%%%%%%%%%%%%%%%%%%%%%%%%%%%%%%%%%%%%%%%%%%%%%%%%%%%%%%%%%%%%%%%%%%%%%%%%%%%%%%%
% Fachwörter starting here
% use g for doubles entries in Abkürzungen und Fachwörter
%%%%%%%%%%%%%%%%%%%%%%%%%%%%%%%%%%%%%%%%%%%%%%%%%%%%%%%%%%%%%%%%%%%%%%%%%%%%%%%%%%%%%%%%%%%%%%%%%%%
\newglossaryentry{annotationen}{
  name = {Annotationen},
  sort = {Annotationen},
  description = {Tupel aus Label und Segmentmarkierung. Die Markierung bezieht sich auf einen Auschnitt innerhalb eines ungeschnittenen Videos und das Label gibt eine in diesem Ausschnitt stattfindende Aktion an.}
}
\newglossaryentry{backpropagation}{
  name = {Backpropagation},
  sort = {Backpropagation},
  description = {}
}
\newglossaryentry{bottleneck}{
  name = {Bottleneck},
  sort = {Bottleneck},
  description = {}
}
\newglossaryentry{channel}{
  name = {Kanal},
  sort = {Kanal},
  plural = {Kanäle},
  description = {}
}
\newglossaryentry{clip}{
  name = {Clip},
  sort = {Clip},
  description = {Ausschnitt fester Länge aus einem ungeschnittenen Video}
}
\newglossaryentry{dataset}{
  name = {Datenset},
  sort = {Datenset},
  description = {Datentupel aus Beispiel und Label}
}
\newglossaryentry{deep-learning}{
  name ={Deep-Learning},
  sort = {Deep Learning},
  description = {}
}
\newglossaryentry{neural-nets}{
  name ={Neuronale Netze},
  sort = {Neuronale Netze},
  description = {}
}
\newglossaryentry{feature}{
  name = {Feature},
  sort = {Feature},
  description = {todo: The key point for low level feature are corners, edges, blobs or contours while high level feature is
more holistic like the structured information related to the action being taken}
}
\newglossaryentry{feature-map}{
  name = {Feature Map},
  sort = {Feature Map},
  description = {todo: }
}
\newglossaryentry{frame}{
  name = {Frame},
  sort = {Frame},
  description = {Einzelnes Bild zu einem festen Zeitpunkt innerhalb eines Videos oder Clips}
}
\newglossaryentry{hyperparameter}{
  name = {Hyperparameter},
  sort = {Hyperparameter},
  description = {todo: A hyperparameter is a property of a learning algorithm, usually (but not always) having a numerical value.
That value influences the way the algorithm works and is not learned from the algorithm itself.}
}
\newglossaryentry{kernel}{
  name = {Kernel},
  sort = {Kernel},
  description = {todo:}
}
\newglossaryentry{label}{
  name = {Label},
  sort = {Label},
  description = {todo: a member of a finite set of classes}
}
\newglossaryentry{layer}{
  name = {Layer},
  sort = {Layer},
  description = {Schicht aus Knoten innerhalb eines Neuronalen Netzes}
}
\newglossaryentry{optical-flow}{
  name = {Optischer Fluss},
  sort = {Optischer Fluss},
  description = {Der optische Fluss einer Bildsequenz ist das Vektorfeld der in die Bildebene projizierten Geschwindigkeit von sichtbaren Punkten des Objektraumes im Bezugssystem der Abbildungsoptik}
}
\newglossaryentry{overfitting}{
  name = {Overfitting},
  sort = {Overfitting},
  description = {Überanpassung der Netz-Parameter an die Trainingsdaten:
    Das Netz hat während des Trainings zu viele Freiheitsgrade, sodass es nicht korrekt abstrahiert, sondern beginnt Beispiele auswendig zu lernen.}
}
\newglossaryentry{pooling}{
  name = {Pooling},
  sort = {Pooling},
  description = {todo:}
}
\newglossaryentry{rgb}{
  name ={RGB},
  sort = {RGB},
  description = {Additives Farbmodell zur Kodierung von Farbwerten aus drei Lichtkanälen (rot, grün und blau)}
}
\newglossaryentry{sample}{
  name = {Sample},
  sort = {Sample},
  description = {Datensatz aus Clip und Labels $(x_i, y_i)$}
}
\newglossaryentry{training}{
  name = {Training},
  sort = {Training},
  description = {}
}
\newglossaryentry{transfer-learning}{
  name = {Transfer Learning},
  sort = {Transfer Learning},
  description = {}
}
\newglossaryentry{vanish-gradient}{
  name = {Vanishing Gradient Problem},
  sort = {Vanishing Gradient Problem},
  description = {}
}


%%%%%%%%%%%%%%%%%%%%%%%%%%%%%%%%%%%%%%%%%%%%%%%%%%%%%%%%%%%%%%%%%%%%%%%%%%%%%%%%%%%%%%%%%%%%%%%%%%%
% Acronyms starting here
% Anmerkung: Nicht wundern - selbe Anfangsbuchstaben rücken näher zusammen im Verzeichnis
% http://tug.ctan.org/tex-archive/macros/latex/contrib/glossaries/glossariesbegin.html#sec:xr
%%%%%%%%%%%%%%%%%%%%%%%%%%%%%%%%%%%%%%%%%%%%%%%%%%%%%%%%%%%%%%%%%%%%%%%%%%%%%%%%%%%%%%%%%%%%%%%%%%%

\newacronym{api}{API}{Application Programming Interface}
\newacronym{bovw}{BOVW}{Bag Of Visual Words}
\newacronym{cnn}{CNN}{Convolutional Neural Network}
\newacronym{csv}{CSV}{Comma-Separated Values}
\newacronym{dnn}{DNN}{Deep Neural Network}
\newacronym{flops}{FLOPS}{Floating Point Operations per Second}
\newacronym{fps}{Fps}{Frames per second}
\newacronym{gpu}{GPU}{Graphics Processing Unit}
\newacronym{gui}{GUI}{Graphical User Interface}
\newacronym{har}{HAR}{Human Action Recognition}
\newacronym{idt}{iDT}{improved Dense Trajectories}
\newacronym{json}{JSON}{JavaScript Object Notation}
\newacronym[see={[Glossar:]{training}}]{ml}{ML}{Machine Learning}
\newacronym{mlp}{MLP}{Multi-Layer Perceptron}
\newacronym{mse}{MSE}{Mean squared error}
\newacronym{ocr}{OCR}{Object Character Recognition}
\newacronym{pca}{PCA}{Hauptkomponentenanalyse}
\newacronym{pes}{PES}{Packetized elementary stream}
\newacronym{pts}{PTS}{Presentation timestamps}
\newacronym{relu}{ReLU}{Rectified Linear Unit}
\newacronym{rnn}{RNN}{Recurrent Neural Network}
\newacronym{sbod}{SBOD}{StatsBomb Open Data}
\newacronym{svm}{SVM}{Support Vector Machine}
\newacronym[see={[Kapitel:]{sec:temporal-action-detection}}]{tsn}{TSN}{Temporal Segment Network}
\newacronym{url}{URL}{Uniform Resource Locator}

%%%%%%%%%%%%%%%%%%%%%%%%%%%%%%%%%%%%%%%%%%%%%%%%%%%%%%%%%%%%%%%%%%%%%%%%%%%%%%%%%%%%%%%%%%%%%%%%%%%
% Symbols starting here
%%%%%%%%%%%%%%%%%%%%%%%%%%%%%%%%%%%%%%%%%%%%%%%%%%%%%%%%%%%%%%%%%%%%%%%%%%%%%%%%%%%%%%%%%%%%%%%%%%%

\newglossaryentry{tld:A}{%
	type=symbols,
	name={A},
	description={Anzahl möglicher Aktionsklassen (Labels)}
}
\newglossaryentry{tld:C}{%
	type=symbols,
	name={C},
	description={Anzahl Kanäle}
}
\newglossaryentry{tld:D}{%
	type=symbols,
	name={D},
	description={Set aus Datensamples}
}
\newglossaryentry{tld:S}{%
	type=symbols,
	name={S},
	description={Räumliche Auflösung}
}
\newglossaryentry{tld:T}{%
	type=symbols,
	name={T},
	description={Anzahl der Frames, zeitliche Auflösung}
}


\newglossaryentry{tld:b}{%
	type=symbols,
	name={b},
	description={Bias eines Layers}
}
\newglossaryentry{tld:d}{%
	type=symbols,
	name={d},
	description={Kerneldimension (temporärer Achse)}
}
\newglossaryentry{tld:i}{%
	type=symbols,
	name={i},
	description={Index eines Samples}
}
\newglossaryentry{tld:k}{%
	type=symbols,
	name={k},
	description={Kerneldimension (räumliche Achsen)}
}
\newglossaryentry{tld:l}{%
	type=symbols,
	name={l},
	description={Layer-Index}
}
\newglossaryentry{tld:r}{%
	type=symbols,
	name={r},
	description={Verhältnis von Samples einer Klasse zu der Gesamtzahl an Samples}
}
\newglossaryentry{tld:u}{%
	type=symbols,
	name={u},
	description={Zellenindex eines Neurons innerhalb eines Layers}
}
\newglossaryentry{tld:w}{%
	type=symbols,
	name={w},
	description={Gewicht}
}
\newglossaryentry{tld:x}{%
	type=symbols,
	name={x},
	description={Input-Vektor $x \in \mathbb{R}^{(C \times T \times S \times S)}$, enthält Clip eines Samples}
}
\newglossaryentry{tld:y}{%
	type=symbols,
	name={y},
	description={Output-One-Hot-Vektor $y \in \{0, 1\}^A$ enthält Labels eines Samples}
}

\newglossaryentry{tld:Delta}{%
	type=symbols,
	name={\Delta},
	description={Dauer eines Clips (Clip-Abdeckung)}
}
\newglossaryentry{tld:sigma}{%
	type=symbols,
	name={$\sigma$},
	description={Aktivierungsfunktion}
}
\newglossaryentry{tld:gamma-tau}{%
	type=symbols,
	name={\gamma_\tau},
	description={Samplingrate: $\tau = 1$ entspricht Original-Framerate}
}
\newglossaryentry{tld:gamma-lambda}{%
	type=symbols,
	name={$\Lambda$},
	description={temporal coverage}
}
\newglossaryentry{tld:Psi}{%
	type=symbols,
	name={\Psi},
	description={Menge der Scores pro gesampleten Clips}
}
\newglossaryentry{tld:theta}{%
	type=symbols,
	name={$\theta$},
	description={Grenzwert}
}
\newglossaryentry{tld:Theta}{%
	type=symbols,
	name={\Theta},
	description={Obergrenze}
}
