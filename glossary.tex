%%%%%%%%%%%%%%%%%%%%%%%%%%%%%%%%%%%%%%%%%%%%%%%%%%%%%%%%%%%%%%%%%%%%%%%%%%%%%%%%%%%%%%%%%%%%%%%%%%%
% Fachwörter starting here
% use g for doubles entries in Abkürzungen und Fachwörter
%%%%%%%%%%%%%%%%%%%%%%%%%%%%%%%%%%%%%%%%%%%%%%%%%%%%%%%%%%%%%%%%%%%%%%%%%%%%%%%%%%%%%%%%%%%%%%%%%%%
\newglossaryentry{annotationen}{
  name = {Annotationen},
  sort = {Annotationen},
  description = {
        Tupel aus Videoverweis, Label und Segmentmarkierung.
        Die Markierung bezieht sich auf einen Ausschnitt innerhalb des Videos und das Label gibt eine in diesem Ausschnitt stattfindende Aktion an.
        Annotationen werden in einer Datenbank gespeichert und sind Grundlage eines Datensets.
    }
}
\newglossaryentry{backbone}{
  name = {Backbone},
  sort = {Backbone},
  description = {
        DNN-Architektur, die in eine übergeordnete Architektur eingebettet wird.
    }
}
\newglossaryentry{batch}{
  name = {Batch},
  sort = {Batch},
  description = {
        Menge an Samples die innerhalb des Trainings parallel verarbeitet werden.
    }
}
\newglossaryentry{bottleneck}{
  name = {Bottleneck},
  sort = {Bottleneck},
  description = {
    Optionaler Block-Typ im Residual Learning zur Verdichtung von Informationen.
    }
}
\newglossaryentry{channel}{
  name = {Kanal},
  sort = {Kanal},
  plural = {Kanäle},
  description = {
    Erste Dimension eines Input-Clips \bzw einer Feature-Map.
    Die Kanäle eines Input-Clips ergeben sich aus den Farbkanälen des Bilds.
    Kanäle der Feature-Maps entsprechen der Anzahl verwendeter Kernels des Layers.
    }
}
\newglossaryentry{clip}{
  name = {Clip},
  sort = {Clip},
  description = {
        Input eines Samples.
        Clips innerhalb eines Datensets haben immer eine feste Dimension und repräsentieren Segmente ungeschnittener Videos.
    }
}
\newglossaryentry{dataset}{
  name = {Datenset},
  sort = {Datenset},
  description = {
        Menge aus Samples zum Training, zur Validierung oder zum Testen.
    }
}
\newglossaryentry{feature}{
  name = {Feature},
  sort = {Feature},
  description = {
        Komprimierte Repräsentation, die als Zwischenergebnis für weiterführende Aufgaben (wie Klassifizierung) genutzt wird.
    }
}
\newglossaryentry{feature-map}{
  name = {Feature Map},
  sort = {Feature Map},
  description = {
        Features, die pro Batch und pro Layer innerhalb des Trainings erhoben werden.
    }
}
\newglossaryentry{frame}{
  name = {Frame},
  sort = {Frame},
  description = {
        Einzelnes Bild innerhalb eines Clips.
    }
}
\newglossaryentry{kernel}{
  name = {Kernel},
  sort = {Kernel},
  description = {Faltungsmatrix im Zusammenhang mit CNNs.}
}
\newglossaryentry{label}{
  name = {Label},
  sort = {Label},
  description = {
        Output eines Samples.
        Ein Label im Kontext von Action Recognition entspricht einer Menge von Aktionsklassen.
    }
}
\newglossaryentry{layer}{
  name = {Layer},
  sort = {Layer},
  description = {
        Schicht innerhalb eines Neuronalen Netzes
    }
}
\newglossaryentry{pooling}{
  name = {Pooling},
  sort = {Pooling},
  description = {
        Technik zur Komprimierung von Feature-Maps innerhalb von CNNs
    }
}
\newglossaryentry{rgb}{
  name ={RGB},
  sort = {RGB},
  description = {
        Additives Farbmodell zur Kodierung von Farbwerten aus drei Lichtkanälen (rot, grün und blau)
    }
}
\newglossaryentry{sample}{
  name = {Sample},
  sort = {Sample},
  description = {
        Tupel aus Input-Clip und Output-Label $(x_i, y_i)$.
        Samples werden in Datensets gespeichert.
    }
}
\newglossaryentry{scores}{
  name = {Scores},
  sort = {Scores},
  description = {
        Output eines Neuronalen Netzes.
        Scores repräsentieren im Zusammenhang mit einer Action Recognition Erkennungswahrscheinlichkeiten pro Aktionsklasse.
    }
}
\newglossaryentry{training}{
  name = {Training},
  sort = {Training},
  description = {
        Optimierungsprozess, in dem die Gewichte eines Neuronalen Netzes an die Samples des Trainingssets angepasst werden.
    }
}
\newglossaryentry{transfer-learning}{
  name = {Transfer Learning},
  sort = {Transfer Learning},
  description = {
        Die weitere Optimierung eines vortrainierten Modells anhand neuer Datensätze.
    }
}


%%%%%%%%%%%%%%%%%%%%%%%%%%%%%%%%%%%%%%%%%%%%%%%%%%%%%%%%%%%%%%%%%%%%%%%%%%%%%%%%%%%%%%%%%%%%%%%%%%%
% Acronyms starting here
% Anmerkung: Nicht wundern - selbe Anfangsbuchstaben rücken näher zusammen im Verzeichnis
% http://tug.ctan.org/tex-archive/macros/latex/contrib/glossaries/glossariesbegin.html#sec:xr
%%%%%%%%%%%%%%%%%%%%%%%%%%%%%%%%%%%%%%%%%%%%%%%%%%%%%%%%%%%%%%%%%%%%%%%%%%%%%%%%%%%%%%%%%%%%%%%%%%%

\newacronym{api}{API}{Application Programming Interface}
\newacronym{bovw}{BOVW}{Bag Of Visual Words}
\newacronym{cnn}{CNN}{Convolutional Neural Network}
\newacronym{csv}{CSV}{Comma-Separated Values}
\newacronym{dnn}{DNN}{Deep Neural Network}
\newacronym{flops}{FLOPS}{Floating Point Operations per Second}
\newacronym{fps}{Fps}{Frames per second}
\newacronym{gpu}{GPU}{Graphics Processing Unit}
\newacronym{gui}{GUI}{Graphical User Interface}
\newacronym{har}{HAR}{Human Action Recognition}
\newacronym{idt}{iDT}{improved Dense Trajectories}
\newacronym{json}{JSON}{JavaScript Object Notation}
\newacronym{ml}{ML}{Machine Learning}
\newacronym{mlp}{MLP}{Multi-Layer Perceptron}
\newacronym{map}{mAP}{mean Average Precision}
\newacronym{ocr}{OCR}{Object Character Recognition}
\newacronym{pca}{PCA}{Hauptkomponentenanalyse}
\newacronym{pes}{PES}{Packetized elementary stream}
\newacronym{pts}{PTS}{Presentation timestamps}
\newacronym{relu}{ReLU}{Rectified Linear Unit}
\newacronym{rnn}{RNN}{Recurrent Neural Network}
\newacronym{sbod}{SBOD}{StatsBomb Open Data}
\newacronym{svm}{SVM}{Support Vector Machine}
\newacronym{tsn}{TSN}{Temporal Segment Network}
\newacronym{url}{URL}{Uniform Resource Locator}

%%%%%%%%%%%%%%%%%%%%%%%%%%%%%%%%%%%%%%%%%%%%%%%%%%%%%%%%%%%%%%%%%%%%%%%%%%%%%%%%%%%%%%%%%%%%%%%%%%%
% Symbols starting here
%%%%%%%%%%%%%%%%%%%%%%%%%%%%%%%%%%%%%%%%%%%%%%%%%%%%%%%%%%%%%%%%%%%%%%%%%%%%%%%%%%%%%%%%%%%%%%%%%%%

\newglossaryentry{tld:A}{%
	type=symbols,
	name={A},
	description={Anzahl möglicher Aktionsklassen (Labels)}
}
\newglossaryentry{tld:C}{%
	type=symbols,
	name={C},
	description={Anzahl Kanäle}
}
\newglossaryentry{tld:D}{%
	type=symbols,
	name={D},
	description={Datenset, Menge aus Samples}
}
\newglossaryentry{tld:S}{%
	type=symbols,
	name={S},
	description={Räumliche Auflösung in Pixeln}
}
\newglossaryentry{tld:T}{%
	type=symbols,
	name={T},
	description={Zeitliche Auflösung in Frames}
}
\newglossaryentry{tld:a}{%
	type=symbols,
	name={a},
	description={Aktionsklasse}
}
\newglossaryentry{tld:x}{%
	type=symbols,
	name={x},
	description={Input-Vektor (Clip) $x \in \mathbb{R}^{(\gls{tld:C} \times \gls{tld:T} \times \gls{tld:S} \times \gls{tld:S})}$}
}
\newglossaryentry{tld:y}{%
	type=symbols,
	name={y},
	description={Output-One-Hot-Vektor (Label) $y \in \{0, 1\}^A$}
}

\newglossaryentry{tld:Delta}{%
	type=symbols,
	name={$\Delta$},
	description={Zeitkontext eines Clips}
}
\newglossaryentry{tld:tau}{%
	type=symbols,
	name={$\tau$},
	description={Sampling-Zeitschritt: $\tau = 1$ entspricht Original-Framerate von 25 Fps}
}
\newglossaryentry{tld:Theta}{%
	type=symbols,
	name={$\Theta$},
	description={Obergrenze verwendeter Samples pro Klasse und Epoche}
}
